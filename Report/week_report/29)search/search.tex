\documentclass{article}
\usepackage{CJK}
% Change "article" to "report" to get rid of page number on title page
\usepackage{amsmath,amsfonts,amsthm,amssymb}
\usepackage{setspace}
\usepackage{Tabbing}
\usepackage{fancyhdr}
\usepackage{lastpage}
\usepackage{extramarks}
\usepackage{chngpage}
\usepackage{soul,color}
\usepackage{graphicx,float,wrapfig}
\usepackage{multirow}
\usepackage{enumerate}
\usepackage{comment} 
% In case you need to adjust margins:
\topmargin=-0.45in      %
\evensidemargin=0in     %
\oddsidemargin=0in      %
\textwidth=6.5in        %
\textheight=9.0in       %
\headsep=0.25in         %

% Homework Specific Information
\newcommand{\hmwkTitle}{Combinatoric Optimization Problem}
\newcommand{\hmwkClass}{}
\newcommand{\hmwkAuthorName}{Donglai\ Wei}


% Setup the header and footer
\pagestyle{fancy}                                                       %
\lhead{\hmwkAuthorName}                                                 %
\rhead{\firstxmark}                                                     %
\lfoot{\lastxmark}                                                      %
\cfoot{}                                                                %
\rfoot{Page\ \thepage\ of\ \pageref{LastPage}}                          %
\renewcommand\headrulewidth{0.4pt}                                      %
\renewcommand\footrulewidth{0.4pt}                                      %

% This is used to trace down (pin point) problems
% in latexing a document:
%\tracingall

%%%%%%%%%%%%%%%%%%%%%%%%%%%%%%%%%%%%%%%%%%%%%%%%%%%%%%%%\begin{enumerate}

% Some tools
\newcommand{\enterProblemHeader}[1]{\nobreak\extramarks{#1}{#1 continued on next page\ldots}\nobreak%
                                    \nobreak\extramarks{#1 (continued)}{#1 continued on next page\ldots}\nobreak}%
\newcommand{\exitProblemHeader}[1]{\nobreak\extramarks{#1 (continued)}{#1 continued on next page\ldots}\nobreak%
                                   \nobreak\extramarks{#1}{}\nobreak}%

\newlength{\labelLength}
\newcommand{\labelAnswer}[2]
  {\settowidth{\labelLength}{#1}%
   \addtolength{\labelLength}{0.25in}%
   \changetext{}{-\labelLength}{}{}{}%
   \noindent\fbox{\begin{minipage}[c]{\columnwidth}#2\end{minipage}}%
   \marginpar{\fbox{#1}}%

   % We put the blank space above in order to make sure this
   % \marginpar gets correctly placed.
   \changetext{}{+\labelLength}{}{}{}}%

\setcounter{secnumdepth}{0}
\newcommand{\homeworkProblemName}{}%
\newcounter{homeworkProblemCounter}%
\newenvironment{homeworkProblem}[1][Problem \arabic{homeworkProblemCounter}]%
  {\stepcounter{homeworkProblemCounter}%
   \renewcommand{\homeworkProblemName}{#1}%
   \section{\homeworkProblemName}%
   \enterProblemHeader{\homeworkProblemName}}%
  {\exitProblemHeader{\homeworkProblemName}}%

\newcommand{\problemAnswer}[1]
  {\noindent\fbox{\begin{minipage}[c]{\columnwidth}#1\end{minipage}}}%

\newcommand{\problemLAnswer}[1]
  {\labelAnswer{\homeworkProblemName}{#1}}

\newcommand{\homeworkSectionName}{}%
\newlength{\homeworkSectionLabelLength}{}%
\newenvironment{homeworkSection}[1]%
  {% We put this space here to make sure we're not connected to the above.
   % Otherwise the changetext can do funny things to the other margin

   \renewcommand{\homeworkSectionName}{#1}%
   \settowidth{\homeworkSectionLabelLength}{\homeworkSectionName}%
   \addtolength{\homeworkSectionLabelLength}{0.25in}%
   \changetext{}{-\homeworkSectionLabelLength}{}{}{}%
   \subsection{\homeworkSectionName}%
   \enterProblemHeader{\homeworkProblemName\ [\homeworkSectionName]}}%
  {\enterProblemHeader{\homeworkProblemName}%

   % We put the blank space above in order to make sure this margin
   % change doesn't happen too soon (else \sectionAnswer's can
   % get ugly about their \marginpar placement.
   \changetext{}{+\homeworkSectionLabelLength}{}{}{}}%

\newcommand{\sectionAnswer}[1]
  {% We put this space here to make sure we're disconnected from the previous
   % passage

   \noindent\fbox{\begin{minipage}[c]{\columnwidth}#1\end{minipage}}%
   \enterProblemHeader{\homeworkProblemName}\exitProblemHeader{\homeworkProblemName}%
   \marginpar{\fbox{\homeworkSectionName}}%

   % We put the blank space above in order to make sure this
   % \marginpar gets correctly placed.
   }%

%%%%%%%%%%%%%%%%%%%%%%%%%%%%%%%%%%%%%%%%%%%%%%%%%%%%%%%%%%%%%



%%%%%%%%%%%%%%%%%%%%%%%%%%%%%%%%%%%%%%%%%%%%%%%%%%%%%%%%%%%%%
% Make title
\title{\vspace{0.3in}\textmd{\textbf{\hmwkTitle}}}
\date{2011.1.16}
\author{\textbf{\hmwkAuthorName}}
%%%%%%%%%%%%%%%%%%%%%%%%%%%%%%%%%%%%%%%%%%%%%%%%%%%%%%%%%%%%%

\begin{document}
\begin{spacing}{1.1}
\maketitle


\section{0)Set Up:}
\begin{CJK}{UTF8}{gbsn}
中餐馆有K道菜供选择,共J家连锁店,顾客可分成W种(e.g.年龄)\\
每家餐馆中,每桌只有一道菜(k相同),人种不限(w任意)\\
\end{CJK}

{\bf \emph{Constants:}}\\
$\alpha,\gamma,\lambda$\\

{\bf \emph{Variable:}}\\
\begin{CJK}{UTF8}{gbsn}
$\{n_{jk}^{w}\}$:在第j家连锁店 $j=1,2,...,J$,点了第k道菜$k=1,2,...,K$的第w种顾客的人数$w=1,2,...,W$
\end{CJK}
{\bf \emph{Statistics:}}\\
$n_{jk}=\sum_{w=1}^{W} n_{jk}^{w}$\\ 
$n_{j.}=\sum_{k=1}^{K} n_{jk}$ \\ \\
\begin{CJK}{UTF8}{gbsn}
$\{m_{jk}\}$:在第j家连锁店 $j=1,2,...,J$,点了第k道菜$k=1,2,...,K$的桌子数量 \\
\end{CJK}
$m_{j.}=\sum_{k=1}^{K} m_{jk}$\\
$m_{..}=\sum_{j=1}^{J} m_{j.}$\\

{\bf \emph{Linear Constraints:}}\\
$\sum_{k=1}^{K} n_{jk}^{w}=n_{j.}^{w}$, for each w\\ \\

{\bf \emph{Nonlinear Object Funciton:}}
\begin{eqnarray*}
F
&=&\sum_{j=1}^{J} \{log \frac{\Gamma(\alpha)}{\Gamma(n_{j..}+\alpha)}+\sum_{k=1}^{K}[log(\Gamma(n_{jk})+log \alpha
]\} \\
&+&log \frac{\Gamma(\gamma)}{\Gamma(m_{..}+\gamma)}+\sum_{k=1}^{K} [log(\frac{\Pi_{w=1}^{W}\Gamma(\lambda+n_{..k}^{w})}{\Gamma(n_{..k}+W\lambda_{0})})+log(\frac{\Gamma(W\lambda)}{\Gamma(\lambda)^{W}})
+log(\Gamma(m_{.k})+log \gamma]\\ 
\end{eqnarray*}

\emph{$\Gamma$: gamma function,$\Gamma(n)=(n-1)! $ for integers}

{\bf \emph{GOAL:}}\\
Clustering problem: find the partition $n_{jk}^{w}$ that maximizes F

\section{2)Modern Heuristic:}
1) relaxed linear programming doesn't seem to fit\\
2) convex optimization doesn't either, though F is a convex function. (The derivative of gamma function is digamma function, which is hard for analytic or numerical)\\
3) So, we choose modern heuristic methods to {\bf SEARCH } for $n_{jk}^{w}$:\\ \\
i) Big Move: since we have 3 coordinates(j,k,w), we can change only one j or k or w\\
ii)Small Move: within the big move above, we can local move,split move and merge move 
\section{3)Additional:}
i) It has some probability meanings and we can use Gibbs sampling and K-means++ for randomization \\
ii) Instead of searching the partition $n_{jk}^{w}$, we can try to search configuration $k_{ji}$, the dish that the i th customer in the j th restaurant is served,
which is easier to search.
\end{spacing}
\end{document}

%%%%%%%%%%%%%%%%%%%%%%%%%%%%%%%%%%%%%%%%%%%%%%%%%%%%%%%%%%%%%
