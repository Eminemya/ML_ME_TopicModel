\documentclass{article}
% Change "article" to "report" to get rid of page number on title page
\usepackage{amsmath,amsfonts,amsthm,amssymb}
\usepackage{setspace}
\usepackage{Tabbing}
\usepackage{fancyhdr}
\usepackage{lastpage}
\usepackage{extramarks}
\usepackage{chngpage}
\usepackage{soul,color}
\usepackage{graphicx,float,wrapfig}
\usepackage{multirow}
\usepackage{enumerate}
% In case you need to adjust margins:
\topmargin=-0.45in      %
\evensidemargin=0in     %
\oddsidemargin=0in      %
\textwidth=6.5in        %
\textheight=9.0in       %
\headsep=0.25in         %

% Homework Specific Information
\newcommand{\hmwkTitle}{Week 7 report}
\newcommand{\hmwkClass}{}
\newcommand{\hmwkAuthorName}{Donglai\ Wei}


% Setup the header and footer
\pagestyle{fancy}                                                       %
\lhead{\hmwkAuthorName}                                                 %
\rhead{\firstxmark}                                                     %
\lfoot{\lastxmark}                                                      %
\cfoot{}                                                                %
\rfoot{Page\ \thepage\ of\ \pageref{LastPage}}                          %
\renewcommand\headrulewidth{0.4pt}                                      %
\renewcommand\footrulewidth{0.4pt}                                      %

% This is used to trace down (pin point) problems
% in latexing a document:
%\tracingall

%%%%%%%%%%%%%%%%%%%%%%%%%%%%%%%%%%%%%%%%%%%%%%%%%%%%%%%%\begin{enumerate}

% Some tools
\newcommand{\enterProblemHeader}[1]{\nobreak\extramarks{#1}{#1 continued on next page\ldots}\nobreak%
                                    \nobreak\extramarks{#1 (continued)}{#1 continued on next page\ldots}\nobreak}%
\newcommand{\exitProblemHeader}[1]{\nobreak\extramarks{#1 (continued)}{#1 continued on next page\ldots}\nobreak%
                                   \nobreak\extramarks{#1}{}\nobreak}%

\newlength{\labelLength}
\newcommand{\labelAnswer}[2]
  {\settowidth{\labelLength}{#1}%
   \addtolength{\labelLength}{0.25in}%
   \changetext{}{-\labelLength}{}{}{}%
   \noindent\fbox{\begin{minipage}[c]{\columnwidth}#2\end{minipage}}%
   \marginpar{\fbox{#1}}%

   % We put the blank space above in order to make sure this
   % \marginpar gets correctly placed.
   \changetext{}{+\labelLength}{}{}{}}%

\setcounter{secnumdepth}{0}
\newcommand{\homeworkProblemName}{}%
\newcounter{homeworkProblemCounter}%
\newenvironment{homeworkProblem}[1][Problem \arabic{homeworkProblemCounter}]%
  {\stepcounter{homeworkProblemCounter}%
   \renewcommand{\homeworkProblemName}{#1}%
   \section{\homeworkProblemName}%
   \enterProblemHeader{\homeworkProblemName}}%
  {\exitProblemHeader{\homeworkProblemName}}%

\newcommand{\problemAnswer}[1]
  {\noindent\fbox{\begin{minipage}[c]{\columnwidth}#1\end{minipage}}}%

\newcommand{\problemLAnswer}[1]
  {\labelAnswer{\homeworkProblemName}{#1}}

\newcommand{\homeworkSectionName}{}%
\newlength{\homeworkSectionLabelLength}{}%
\newenvironment{homeworkSection}[1]%
  {% We put this space here to make sure we're not connected to the above.
   % Otherwise the changetext can do funny things to the other margin

   \renewcommand{\homeworkSectionName}{#1}%
   \settowidth{\homeworkSectionLabelLength}{\homeworkSectionName}%
   \addtolength{\homeworkSectionLabelLength}{0.25in}%
   \changetext{}{-\homeworkSectionLabelLength}{}{}{}%
   \subsection{\homeworkSectionName}%
   \enterProblemHeader{\homeworkProblemName\ [\homeworkSectionName]}}%
  {\enterProblemHeader{\homeworkProblemName}%

   % We put the blank space above in order to make sure this margin
   % change doesn't happen too soon (otherwise \sectionAnswer's can
   % get ugly about their \marginpar placement.
   \changetext{}{+\homeworkSectionLabelLength}{}{}{}}%

\newcommand{\sectionAnswer}[1]
  {% We put this space here to make sure we're disconnected from the previous
   % passage

   \noindent\fbox{\begin{minipage}[c]{\columnwidth}#1\end{minipage}}%
   \enterProblemHeader{\homeworkProblemName}\exitProblemHeader{\homeworkProblemName}%
   \marginpar{\fbox{\homeworkSectionName}}%

   % We put the blank space above in order to make sure this
   % \marginpar gets correctly placed.
   }%

%%%%%%%%%%%%%%%%%%%%%%%%%%%%%%%%%%%%%%%%%%%%%%%%%%%%%%%%%%%%%



%%%%%%%%%%%%%%%%%%%%%%%%%%%%%%%%%%%%%%%%%%%%%%%%%%%%%%%%%%%%%
% Make title
\title{\vspace{0.3in}\textmd{\textbf{\hmwkTitle}}}
\date{2010.10.12}
\author{\textbf{\hmwkAuthorName}}
%%%%%%%%%%%%%%%%%%%%%%%%%%%%%%%%%%%%%%%%%%%%%%%%%%%%%%%%%%%%%

\begin{document}
\begin{spacing}{1.1}
\maketitle

\section{1) Reorganization of ME moves}
\subsection{A) Focus on Dish Config}
Inspired by "Reading Tea Leaves: How Humans Interpret Topic Models":\\
{\bf Focus on the latent space:}\\
1) How well is a certain restaurant explained by the dishes ?$\Rightarrow$ Decompose-Restaurant Move(lower level)\\
2) How representative are the dishes? $\Rightarrow$ Dish Refinement(higher level)\\ \\
So, we are going to do higher-level moves, directly refining dish config.
\subsection{B) Randomness and Efficiency}
For dish config refinement, now we have {\bf Decompose-Dish,Merge-Dish,Local-Dish},which are in decreased order in terms of complexity.\\ \\
0) Instead of doing the move for all dishes, we use a move for one dish at a time so that they are "light" enough to mix better.\\
1) We may need Randomness to choose a move s.t. Decompose-Dish is relatively less frequently.\\
2) In Decompose-Dish, we may want to decompose noiser dishes more frequently.\\
3) In Merge-Dish, we may want to merge smaller dishes more frequently.\\

\subsection{C) Pseudocode}
\begin{alltt}
  \(\%a)Initialization:\)
      Gibbs-Initialization

  \(\%b)Run for Convergence:\)
      While Likelihood doesn't increase any more:          
          Switch(randsample([1,2,3],weight_move))
              case 1:
              \(\%1)Local Word Refinement () :\)
                    Local-Word(randsample(dish))
              case 2:
              \(\%1)Merge Dish Refinement () :\)
                    weight~num of words in the dish
                    Merge-Dish(randsample(dish,weight))
              case 3:
              \(\%1)Decompose Dish Refinement () :\)
                    weight~-Likelihood/num of words in the dish
		    Decompose-Dish(randsample(dish,weight))
	  End
      End
\end{alltt}
\begin{enumerate}[i]
 \item Decompose-Dish:
\begin{enumerate}[a]
 \item In the beginning, Decompose-Restaurant(delete previous restaurant config) is used to get out of stuck given bad dishes.
 \item Later when the dishes become better, Decompose-Table(only delete tables with certain dish) is used for efficiency.
\end{enumerate}
 \item Merge-Dish: It may only be helpful in the beginning, when there tend to be many small dishes.
 \item Local-Word: 
\begin{enumerate}[a]
 \item Can be seen as a bigger version of Local-customer, where we try to reconfig one big "customer" in the dish
 \item A little drift away to Decompose-Word, we can delete the word configuration and resample+search 
\end{enumerate}
\end{enumerate}


\section{2) Gibbs Sampling with FAIR INITIALIZATION}
\section{3) Results on Topic Model}
Here, I run Gibbs Sampling and ME algorithm on a synthetic data. 100 words and 200 documents.
Hungarian Algorithm to match similar topics with a hamming distance metric.
\end{spacing}
\end{document}

%%%%%%%%%%%%%%%%%%%%%%%%%%%%%%%%%%%%%%%%%%%%%%%%%%%%%%%%%%%%%
