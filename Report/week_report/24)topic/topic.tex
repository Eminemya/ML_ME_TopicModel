\documentclass{article}
% Change "article" to "report" to get rid of page number on title page
\usepackage{amsmath,amsfonts,amsthm,amssymb}
\usepackage{setspace}
\usepackage{Tabbing}
\usepackage{fancyhdr}
\usepackage{lastpage}
\usepackage{extramarks}
\usepackage{chngpage}
\usepackage{soul,color}
\usepackage{graphicx,float,wrapfig}
\usepackage{multirow}
\usepackage{enumerate,alltt}
\usepackage{comment} 
% In case you need to adjust margins:
\topmargin=-0.45in      %
\evensidemargin=0in     %
\oddsidemargin=0in      %
\textwidth=6.5in        %
\textheight=9.0in       %
\headsep=0.25in         %

% Homework Specific Information
\newcommand{\hmwkTitle}{Approximation: from Discrete to Continuous}
\newcommand{\hmwkClass}{}
\newcommand{\hmwkAuthorName}{Donglai\ Wei}


% Setup the header and footer
\pagestyle{fancy}                                                       %
\lhead{\hmwkAuthorName}                                                 %
\rhead{\firstxmark}                                                     %
\lfoot{\lastxmark}                                                      %
\cfoot{}                                                                %
\rfoot{Page\ \thepage\ of\ \pageref{LastPage}}                          %
\renewcommand\headrulewidth{0.4pt}                                      %
\renewcommand\footrulewidth{0.4pt}                                      %

% This is used to trace down (pin point) problems
% in latexing a document:
%\tracingall

%%%%%%%%%%%%%%%%%%%%%%%%%%%%%%%%%%%%%%%%%%%%%%%%%%%%%%%%\begin{enumerate}

% Some tools
\newcommand{\enterProblemHeader}[1]{\nobreak\extramarks{#1}{#1 continued on next page\ldots}\nobreak%
                                    \nobreak\extramarks{#1 (continued)}{#1 continued on next page\ldots}\nobreak}%
\newcommand{\exitProblemHeader}[1]{\nobreak\extramarks{#1 (continued)}{#1 continued on next page\ldots}\nobreak%
                                   \nobreak\extramarks{#1}{}\nobreak}%

\newlength{\labelLength}
\newcommand{\labelAnswer}[2]
  {\settowidth{\labelLength}{#1}%
   \addtolength{\labelLength}{0.25in}%
   \changetext{}{-\labelLength}{}{}{}%
   \noindent\fbox{\begin{minipage}[c]{\columnwidth}#2\end{minipage}}%
   \marginpar{\fbox{#1}}%

   % We put the blank space above in order to make sure this
   % \marginpar gets correctly placed.
   \changetext{}{+\labelLength}{}{}{}}%

\setcounter{secnumdepth}{0}
\newcommand{\homeworkProblemName}{}%
\newcounter{homeworkProblemCounter}%
\newenvironment{homeworkProblem}[1][Problem \arabic{homeworkProblemCounter}]%
  {\stepcounter{homeworkProblemCounter}%
   \renewcommand{\homeworkProblemName}{#1}%
   \section{\homeworkProblemName}%
   \enterProblemHeader{\homeworkProblemName}}%
  {\exitProblemHeader{\homeworkProblemName}}%

\newcommand{\problemAnswer}[1]
  {\noindent\fbox{\begin{minipage}[c]{\columnwidth}#1\end{minipage}}}%

\newcommand{\problemLAnswer}[1]
  {\labelAnswer{\homeworkProblemName}{#1}}

\newcommand{\homeworkSectionName}{}%
\newlength{\homeworkSectionLabelLength}{}%
\newenvironment{homeworkSection}[1]%
  {% We put this space here to make sure we're not connected to the above.
   % Otherwise the changetext can do funny things to the other margin

   \renewcommand{\homeworkSectionName}{#1}%
   \settowidth{\homeworkSectionLabelLength}{\homeworkSectionName}%
   \addtolength{\homeworkSectionLabelLength}{0.25in}%
   \changetext{}{-\homeworkSectionLabelLength}{}{}{}%
   \subsection{\homeworkSectionName}%
   \enterProblemHeader{\homeworkProblemName\ [\homeworkSectionName]}}%
  {\enterProblemHeader{\homeworkProblemName}%

   % We put the blank space above in order to make sure this margin
   % change doesn't happen too soon (else \sectionAnswer's can
   % get ugly about their \marginpar placement.
   \changetext{}{+\homeworkSectionLabelLength}{}{}{}}%

\newcommand{\sectionAnswer}[1]
  {% We put this space here to make sure we're disconnected from the previous
   % passage

   \noindent\fbox{\begin{minipage}[c]{\columnwidth}#1\end{minipage}}%
   \enterProblemHeader{\homeworkProblemName}\exitProblemHeader{\homeworkProblemName}%
   \marginpar{\fbox{\homeworkSectionName}}%

   % We put the blank space above in order to make sure this
   % \marginpar gets correctly placed.
   }%

%%%%%%%%%%%%%%%%%%%%%%%%%%%%%%%%%%%%%%%%%%%%%%%%%%%%%%%%%%%%%



%%%%%%%%%%%%%%%%%%%%%%%%%%%%%%%%%%%%%%%%%%%%%%%%%%%%%%%%%%%%%
% Make title
\title{\vspace{0.3in}\textmd{\textbf{\hmwkTitle}}}
\date{2010.8.3}
\author{\textbf{\hmwkAuthorName}}
%%%%%%%%%%%%%%%%%%%%%%%%%%%%%%%%%%%%%%%%%%%%%%%%%%%%%%%%%%%%%

\begin{document}
\begin{spacing}{1.1}
\maketitle
\section{1) Gibbs Sampler LDA}
\subsection{10 Topics}
\begin{alltt}
    'algorithm matrix problem optimal algorithms step state'
    'learning set error training probability number rate'
    'function functions space points point number class'
    'data information noise gaussian case linear analysis'
    'vector representation pattern figure patterns structure memory'
    'network networks neural input units'
    'model models parameters order figure'
    'neural neurons neuron fig time networks'
    'image field cells local visual cell figure'
    'time figure system input current output signal'

\end{alltt}

\subsection{20 Topics}
\begin{alltt}
    'algorithm matrix optimal step'
    'learning state rate'
    'set error training probability'
    'function functions'
    'information noise gaussian linear'
    'data based process'
    'order problem parameters methods'
    'analysis values case random'
    'vector representation structure processing level simple features'
    'number large size'
    'image local point'
    'pattern patterns memory recognition performance single'
    'figure shown shows'
    'model'
    'network networks'
    'time system'
    'units input unit weights layer'
    'input output current inputs control'
    'neurons neuron fig'
    'field cells visual'
\end{alltt}

\section{2) Gibbs Sampler HDP}
It gives 178 topics and we here display the biggest 20 topics.
\subsection{Gibbs CRF Sampler}
\begin{alltt}
  time signal system figure output input processing neural shown information
 figure model shown order neural function results systems paper based
 neural networks number function network functions problems results order case
 network networks figure neural paper systems input time case based
 neural time neuron figure networks cell neurons systems system shown
 model cells figure visual cell input results shown similar single
 function functions space method problem linear set order data number
 time point network neurons system state neural functions neuron large
 problem matrix model results function figure approach algorithms set problems
 figure control system problem time shows systems shown performance space
 image problem noise data shown field small values current high
 weights input weight learning output function neural small layer network
 data function model analysis parameters classification space vector values form
 algorithm function data set point step local problems neural networks
 learning state algorithm algorithms optimal results values function probability class
 network neurons neural input output number state neuron shown fig
 learning error examples set distribution probability training space values networks
 data linear algorithm noise case approach matrix problem analysis information
 model based neural rate data optimal figure analysis paper case
 error results process function noise gaussian methods performance method small
\end{alltt}

\subsection{Gibbs Block Sampler}
It gives 32 topics and we here display the biggest 20 topics.  
\begin{alltt}
 time figure system signal output input neural shown systems visual
 network neural networks input layer weights number figure output net
 model models figure based data structure analysis parameters shown neural
 function functions space linear method distribution set order methods class
 algorithm optimal step problem figure control function space methods state
 cells cell visual field neurons figure model local input shown
 data noise gaussian information distribution parameters linear figure analysis shown
 algorithm matrix problem learning algorithms results set vector function form
 neuron neurons neural networks state network time model systems parameters
 system point time function local networks order neural models network
 values order size number probability state case random set results
 training data learning set case error number neural trained hidden
 learning algorithm algorithms state rate time distribution process probability form
 image visual point local points number based high signal values
 units unit output input hidden weights system figure space representation
 information fig time neural system neurons large performance single task
 vector representation space structure level case shown processing points network
 pattern patterns single figure results systems analysis based order process
 current fig input output shown shows large inputs small cell
 figure recognition feature task performance high features problem representation neural
\end{alltt}


\end{spacing}
\end{document}

%%%%%%%%%%%%%%%%%%%%%%%%%%%%%%%%%%%%%%%%%%%%%%%%%%%%%%%%%%%%%
